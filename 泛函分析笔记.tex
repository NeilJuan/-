\documentclass[11pt]{article}
\usepackage{color}
\usepackage{xcolor}
\usepackage{graphics}
\usepackage{graphicx}
\usepackage{geometry}
\geometry{left=1.5cm,right=1.5cm,top=2.5cm,bottom=2.5cm}
\usepackage{latexsym,bm}
\usepackage{algorithm}
\usepackage{algorithmicx}
\usepackage[noend]{algpseudocode}
\usepackage{amsmath}
\usepackage{amssymb}%花体字母加粗
\usepackage{mathrsfs}%花体字母
\usepackage{caption}
\usepackage{amsmath}
\usepackage{CJK}
\usepackage{fancyhdr}
\pagestyle{fancy}
\lhead{} %左上头脚注
\chead{} %中间上头脚注
\rhead{\bfseries University of Chinese Academy of Science}%右边上头 
\lfoot{TianyangZhang} %左下头
\cfoot{201728017419004}%中下头
\rfoot{\thepage} %右下头


\begin{document}
\begin{CJK*}{UTF8}{gbsn}

%Title-------------------------------------------
\begin{titlepage}
	
\title{\protect \includegraphics[width=0.5in]{/Users/genius/Desktop/Pictures/UCAS.png} \Huge \textbf{泛函分析笔记}}

\author{张天阳\thanks{学号201728017419004}\\
University of Chinese Academy of Science}
\maketitle
%end title----------------------------------------
\end{titlepage}



%目录------------------------------------------
\tableofcontents

\pagebreak
%end 目录------------------------------------------

%-----------------正文-----------
\part{开闭集与有界集}
\section{度量空间}

\section{开闭集}
\subsection{邻域定义:}
\subsubsection{球邻域}
\begin{enumerate}
	\item 开球(r邻域):$B(\bm x_0,r)=\{\bm x \in X,~d(\bm x,\bm x_0)<r \}$
	\item 闭球:$\overline{B}(\bm x_0,r)=\{\bm x \in X,~d(\bm x,\bm x_0)\leq r \}$
	\item 球面:$S(\bm x_0,r)=\{\bm x \in X,~d(\bm x,\bm x_0)=r \}$
\end{enumerate}
\subsubsection{$\varepsilon$-邻域}
\begin{equation*}
	B(\bm x_0,\varepsilon )=\{\bm x \in X,~d(\bm x,\bm x_0)<\varepsilon \}
\end{equation*}
\subsubsection{邻域}
$\exists ~B(\bm x_0,\varepsilon )\subset M$,则称$M$为$\bm x_0$的一个邻域。
\subsection{开闭区间定义:}
\subsubsection{开区间:}
对于$M\subset X$是开区间,需要满足$\forall \bm x \in M,~\exists~ B(\bm x,\varepsilon )\subset M $。
\subsubsection{闭区间:}
对于$M\subset X$是开区间,需要$M^c=X/M$满足$M^c$是开集(开区间)。
\subsection{内点与有界}
\subsubsection{内点:}
$\bm x_0\in M$是$M$的内点,需要满足:$\exists~ \varepsilon >0 $使得$B(\bm x_0,\varepsilon )\in M$。
\subsubsection{内部:}
$M^c$是$M$的内部,其是所有$M$的内点的集合。

\subsubsection{有界:}
\begin{equation*}
	\exists~r<\infty~,st~M\subset B(\bm x_0,r),~\bm x_0\in X
\end{equation*}
则称$M$有界。

\subsection{定理:}
如果一个集合是开集,其在任意度量下也是开集;闭集同理。\\
一个序列在$d$度量下逼近点$\bm x_0$那么其在$d'$度量下也逼近点$\bm x_0$


\section{例子:}
\subsection{分辨开闭集合,在$(X=R,d(x,y)=|x-y|)$的情况下}
\subsubsection{$M=Q$,有理数集}
\begin{itemize}
	\item 由于任意有理数的任意小$\varepsilon$邻域都有无理数存在,所以其不是开集。
	\item $Q^c$是无理数集,由于无理数的任意小$\varepsilon$邻域都有有理数存在,所以其不是开集。由于$Q^c$不是开集,所以$Q$不是闭集。
\end{itemize}
结论:$Q$为非开非闭。当然无理数集也是非开非闭。
\subsubsection{$M=\phi$,空集}
\begin{itemize}
	\item 由于开集是满足$\forall \bm x \in M,~\exists~ B(\bm x,\varepsilon )\subset M $,而空集中没有点,所以对于其中任意点自然满足。
	\item 由于空集的补集在这里是全集也就是$R$而$R$满足开集的性质,所以空集也是闭集。
\end{itemize}
结论:空集既开又闭。同样全集既开又闭。
\subsubsection{$M=[0,1)$}
\begin{itemize}
	\item 显然M不是开集,因为0点不存在$\varepsilon$邻域使其包含于$M$内。
	\item 又因为其补集中点1也是不存在$\varepsilon$邻域使其包含于$M^c$内。所以也不是闭集。
\end{itemize}
结论该集合非开非闭。此外$(0,1)$开集,$[2,3]$闭集。
\subsection{分辨开闭集合,在$(X=[0,1),d(x,y)=|x-y|)$的情况下}
\subsubsection{$M=[0,1)$}
\begin{itemize}
	\item 由于	$B(\bm x_0,\varepsilon )=\{\textcolor{blue}{\bm x \in X},~d(\bm x,\bm x_0)<\varepsilon \}$,所以所有在集合X内的点的$\varepsilon$邻域都包含于X,所以其为开。
	\item 由于空集为开,所以其为闭集合。所以既开又闭。
\end{itemize}
\subsubsection{$M=(0,\frac{1}{2}]$}
\begin{itemize}
	\item 因为点$\frac{1}{2}$不存在$\varepsilon$邻域使其包含于$M$内,所以其不是开集。
	\item 但是其补集$(\frac{1}{2},1)$是开集,所以其是闭集。
\end{itemize}
\pagebreak
\part{导集、可列集、稠密集与可分}
\setcounter{section}{0}
\section{聚点、导集与闭包}
\subsection{集合$M$的聚点}
若$\bm x_0$为聚点,则需要满足:$\forall~B(\bm x_0,\varepsilon ),~\exists~\bm y \in B(\bm x_0,\varepsilon )\cap M$\\
(想象一下直观上就是$\bm x_0$紧贴着$M$,连在一起,但不一定属于$M$可能是$M$的边界之类)
\subsection{集合$M$的导集(聚集)$M'$}
关于集合$M$的所有聚点的集合叫做其导集。\\

\subsection{集合$M$的闭包$\overline{M}$}
$\overline{M}=M\cup M'$

\subsection{命题:度量空间$(X,d),~M\subset X$中$~\overline{M}$一定是闭集。}
证明:\\
$\overline{M}=M\cup M'\Rightarrow (\overline{M})^c=M^c\cap {(M')}^c$\\
$\forall \bm x\in (\overline{M})^c,\Rightarrow \bm x\in M^c,~\bm x \in {(M')}^c$~beacuse:$~\bm x \in {(M')}^c,~\exists B(\bm x,\varepsilon  )\cap M=\phi\Rightarrow \exists ~B(\bm x,\varepsilon  )\subset M^c$\\
又因为:$\exists ~B(\bm x,\varepsilon  )\subset M^c$所以$\forall \bm y \in B(\bm x,\varepsilon  ),~\bm y\notin M$所以$B(\bm x,\varepsilon  )\in {(M')}^c$\\
即:$B(\bm x,\varepsilon  )\in M^c\cap {(M')}^c\Rightarrow ~\overline{M}^c$是开集,所以$\overline{M}$是闭集。

\subsection{定理:度量空间$(X,d),~M$是闭集$\Leftrightarrow ~\overline{M}=M$}
证明:\\
\begin{itemize}
	\item 必要性:由命题1.4显然。
	\item 充分性:由于$\overline{M}=M\cup M'$只要证:$\overline{M}\subset M$即证$M'\subset M$,即证$M^c\subset (M')^c$。\\
	由于$M$闭所以$M^c$开集,所以$\forall \bm x\in M^c~,\exists ~B(\bm x,\varepsilon )\subset M^c$所以$\bm x$不属于$M'$即$\bm x\in (M')^c $
\end{itemize}
\section{可列集、可数集、稠密与可分}
\subsection{可列集}
可以与自然数一一对应,~$X\rightarrow N$
\subsection{可数集}
若集合$X$是可列集或者有限集(集合内元素个数有限)则称其为可数集。
\subsection{可数集的例子,与不可数集的例子}
\begin{itemize}
	\item 可数集:自然数集、偶数集、\{0,1,0.12317,$\sqrt{3}$\}。
	\item 不可数集:$(0,1)$、$R$、无理数
\end{itemize}


\subsection{命题:可列个可列集的并一定是可列集,$\bigcup^{\infty}_{n=1} A_n=A,~A$是可列集}
证明:沿着对角线数,可以与自然数一一对应。

\subsection{例:有理数集是可列集}
证明:有理数可以表述为$\frac{q}{p};~q,p\in N$,所以将其写作$(q,p)$的形式去数。\\
当$q=1$时,$p$为可列个;当$q=1$时,$p$为可列个,...,当$q=n$时,$p$为可列个,...
可列个可列集的并集是可列集。


\subsection{$X$的稠密子集}
$(X,d)$内,当$\overline{M}=X$则称M是X中的稠密子集。

\subsection{$X$的可分性}
若$(X,d)$内存在稠密子集则称其为可分的。
\section{例:可分性判别}

\subsection{$(R,d)$}
因为任意无理数的任意小邻域都有有理数,所以$Q'=Q^c$所以$\overline{Q}=Q^c\cup Q=R$,所以Q是稠密集。所以可分。
\subsection{$(l_{p_1},d_{p_1})$可分}
\begin{align*}
	l_{p_1}=\{(x_1,x_2,...,x_n,...),x_i\in R,\sum^{\infty}_{i=1}|x_i|<\infty\}\\
	d_{p_1}(x,y)=\|\bm x-\bm y\|_p
\end{align*}
\subsection{$(l_{p_1},d_{p_2})$不可分}
\begin{align*}
	l_{p_1}=\{(x_1,x_2,...,x_n,...),x_i\in R,\sum^{\infty}_{i=1}|x_i|<\infty\}\\
	d_{p_2}(x,y)=\|\bm x-\bm y\|_\infty
\end{align*}
\subsection{$(l[a,b],d(f,g))$可分}
\begin{align*}
	l[a,b]=\{f~,[a,b]\rightarrow R,~continous\}\\
	d(f,g)= \mathop{max}_{t\in [a,b]}|f(t)-g(t)|
\end{align*}
因为多项式函数是稠密的其可以逼近任意定义在$[a,b]$上连续函数。而有理系数多项式又可以任意精度逼近所有多项式函数。

\pagebreak
\part{拓扑空间、连续与序列}
\setcounter{section}{0}
\section{拓扑空间定义:}
\subsection{集类$\tau$}
\begin{equation}
	\tau = \{ A, A\subset X\}
\end{equation}
\subsection{拓扑空间$(X,\tau)$,与拓扑空间中的开集}
\begin{itemize}
	\item 	如果全集$X$(空间)满足下面三条性质,则称其为拓扑空间。
		\begin{enumerate}
			\item $\phi \in \tau,~X \in \tau$
			\item $U_A\in \tau ,~\bigcup_{i=1}^{n}{U_A}\in \tau ,n~can~be~\infty$
			\item $U_A\in \tau ,~\bigcap_{i=1}^{n}{U_A}\in \tau,~n<\infty $
		\end{enumerate}
	\item 此时称$X$中的元素为一个点,$\tau$中的元素为开集,$(X,\tau)$为拓扑空间。
\end{itemize}

\section{连续性与连续函数定义}
\subsection{连续性}
\begin{equation*}
	f:(X,d_1)\rightarrow(Y,d_2)
\end{equation*}
$\forall \bm x_0 \in X$,$\forall \varepsilon >0,~\exists \delta=\delta(\bm x_0,\varepsilon )>0$\\
使得:
\begin{equation*}
	f~B(\bm x_0,\delta )~\subset B(f(\bm x_0),\varepsilon )
\end{equation*}
则称$f$在$\bm x_0$点处连续。\\
\textbf{如果$f$在$X$上连续则称$f$连续。}

\subsection{定理:连续与原像的拉开集等价}
对于任意$Y$空间的开子集$U$其原像$ T^{-1}(U)=\{\bm x\in X~|~T(\bm x)\subset U\}$也是开集。$\Leftrightarrow~T$连续
\textbf{证明:}\\
\begin{enumerate}
	\item "$\Leftarrow$"设T连续:令$ S\subset Y,~S$是开集。\\
	下证:$ T^{-1}(S)$也是开集
	 对于$\forall \bm y\in S$。由于$S$是开集,所以$\exists~B(\bm y,\varepsilon )_{d_2}\subset S$\\
		因为连续:所以存在$~B(\bm x,\varepsilon ),~T~B(\bm x,\varepsilon )\subset B(\bm y,\varepsilon )_{d_2}\subset S$\\
		即$ T~B(\bm x,\varepsilon )\subset S$\\
		即$ B(\bm x,\varepsilon )\subset T^{-1}(S)$
	
	\item "$\Rightarrow$"证明连续:\\
		对于任意$\varepsilon$有$B(T(\bm x_0),\varepsilon ))$是$Y$中的开集,且$T^{-1}B(T(\bm x_0),\varepsilon )$也是开集。\\
		因为$T^{-1}B(\bm x,\varepsilon )$是开集,所以存在$B(\bm x,\delta )\subset T^{-1}B(T(\bm x_0),\varepsilon )$,所以$T~B(\bm x,\delta )\subset B(T(\bm x_0),\varepsilon )$
\end{enumerate}
\begin{figure}[H]
\center
  \includegraphics[width=5in]{/Users/genius/Desktop/FuntionalAnalysis/continuously/continuously.jpeg}
  \caption{连续性}
\end{figure}
\section{序列}
\subsection{定义:序列收敛}
\begin{equation}
	\{x_n\}_{n=1}^{\infty}\subset X
\end{equation}
若存在$x\in X$使得:
\begin{equation*}
	\mathop{lim}_{n\rightarrow \infty}d(x_n,x)=0
\end{equation*}
其为给定一个$\varepsilon$使得存在$N>0$当$n>N时$,$d(x_n,x)<\varepsilon$\\

则称$\{x_n\}$收敛至$x$,记作:
\begin{align*}
	\{x_n\}\rightarrow x\\
	or~~x_n\rightarrow x
\end{align*}
\subsection{直径$\delta(M)$}
	\begin{equation}
		\delta(M)=\sup_{x\subset X,~y\subset Y}d(x,y)
	\end{equation}

\subsection{有界}	
若$\delta(M)<\infty$则$M$有界。\\
要证:$\forall x_0\in X,\exists r>0,M\subset B(x_0,r)$\\
设$y\in M$令$r=d(x_0,y_0)+\delta(M)+1$,因为$d(x,x_0)<d(x,y_0)+d(y_0,x_0)<d(x_0,y_0)+\delta(M)$\\所以$M\subset B(x_0,r)$。
\subsection{性质:若${x_n}$收敛,则其有界,且极限唯一}
\begin{itemize}
	\item 有界:给定一个$\varepsilon=1$使得存在$N>0$当$n>N时$,$d(x_n,x)<\varepsilon$\\
取$a=max\{x_1,x_2,...,x_n\},~\sup_{x\subset X,~y\subset Y}d(x,y)=a+1<\infty$所以有界
\item 唯一:反正假设有$x_n\rightarrow z$,~$\mathop{lim}_{n\rightarrow\infty}d(x,z)\leq d(x_n,z)+d(x_n,x)=0+0=0$所以$x=z$所以唯一。
\end{itemize}
\subsection{性质$x_n\rightarrow x,y_n\rightarrow y,~d(x_n,y_n)\rightarrow d(x,y)$}
证明:\\
$d(x_n,y_n)\leq d(x_n,x)+d(x,y)+d(y_n,y)$\\
$\Rightarrow d(x_n,y_n)-d(x,y)\leq d(x_n,x)+d(y_n,y)$\\
$d(x,y)\leq d(x_n,x)+d(x_n,y_n)+d(y_n,y)$\\
$\Rightarrow d(x,y)-d(x_n,y_n)\leq d(x_n,x)+d(y_n,y)$

所以夹逼定理得到$d(x_n,y_n)\rightarrow d(x,y)$
\subsection{柯西序列}
$(X,d)$度量空间内存在序列$\{x_n\},x_n \in X$满足:\\
对于$\forall \varepsilon >0,~\exists N,~s.t:\forall n,m>N:~~ d(x_n,x_m)<\varepsilon$则称序列$\{x_n\}$为柯西序列。\\(不需要定义收敛点的收敛序列,当其收敛点在空间内则为收敛序列)
\subsection{完备空间}
当空间$M$内的任意柯西序列$\{x_n\} $收敛到$x$,并且$~x\in M$则称空间$M$为完备空间。
\subsection{完备性例子:}
\subsubsection{$(X=(0,1),d=||)$,不是完备空间}
因为$\{X_n\}=\frac{1}{n} $属于这个空间,并且$x_n\rightarrow 0$而$0$不在该空间内。
\subsubsection{$(l_{p_1},d_{p_1})$完备}
\begin{align*}
	l_{p_1}=\{(x_1,x_2,...,x_n,...),x_i\in R,\sum^{\infty}_{i=1}|x_i|<\infty\}\\
	d_{p_1}(x,y)=\|\bm x-\bm y\|_p
\end{align*}
\subsubsection{$(c{a,b},d)$完备}
\begin{align*}
	l[a,b]=\{f~,[a,b]\rightarrow R,~continous\}\\
	d(f,g)= \mathop{max}_{t\in [a,b]}|f(t)-g(t)|
\end{align*}
\subsubsection{有理数空间Q不完备}
因为有理数可以任意逼近无理数,所以不完备。
\subsubsection{$P(a,b),d$不完备}
多项式函数空间不完备,因为多项式函数可以任意精度逼近任意连续函数。
\subsection{定理:任意收敛序列都是柯西序列}
对于任意收敛序列$\{x_n\}$求证$\forall \varepsilon>0,\exists N$满足:$\forall n,m>N$,~$d(x_n,x_m)<\varepsilon$\\
因为序列$\{x_n\} $收敛所以:对于$\frac{\varepsilon}{2}$存在$K$满足:当$n,m>K$,$d(x,x_n)<0.5\varepsilon,d(x_m,x)<0.5\varepsilon$,\\那么:$d(x_n,x_m)<d(x,x_n)+d(x,x_m)<\varepsilon$当$N=K$时。\\
\textcolor{blue}{柯西序列与收敛序列的区别:\\
\begin{enumerate}
	\item 首先收敛序列一定是柯西序列
	\item 柯西序列不一定“收敛”(但一定有极限)原因是其收敛的极限不一定在空间内,除非该空间是个完备空间。比如$\{x_n\}=\frac{1}{n}$在空间$((0,1),d=||)$下是一个柯西序列但是其极限不在该空间内,所以其是不完备,序列也不收敛,但是有极限极限为0不在$(0,1)$内。
	\item 也就是说柯西序列其实是一个有极限(不是无穷)“收敛”的序列,一般用于我们不好搞清收敛极限的时候,因为收敛的定义要那个极限$x,~$$d(x,x_n)<\varepsilon$而且要求极限$x$在该空间内。
\end{enumerate}}
\pagebreak
\part{序列的应用}
\setcounter{section}{0}
\section{定理:$M\subset X, x\in \overline M\Leftrightarrow \exists \{x_n\}\subset M$使得$x_n\rightarrow x$}
\textcolor{blue}{即导集的意义就是里面的点都是某个序列的极限点。}\textcolor{red}{X不需要完备是因为是收敛到x,而不是柯西序列收敛到x。因为是收敛到x极限点x一定在X内}\\
其正向“$\Rightarrow $”$x\in \overline M$,则一定存在$\{x_n\}\subset M$,$x_n\rightarrow x$
\begin{itemize}
	\item 如果$x\in M$那么$\exists \{x_n\}=x $满足条件
	\item 若果$x\in M'$那么$ B(x,\frac{1}{n} )\cap M\neq \phi$,设$x_n\in B(x,\frac{1}{n})$那么$n\rightarrow \infty,~d(x_n,x)=0$。所以找到了一个序列$\{x_n\} $
	\end{itemize}
反向"$\Leftarrow$"任意$\{x_n\}\rightarrow x,~\{x_n\}\subset M$则$x\in \overline M$。
	$\forall \varepsilon>0~,\exists N,~B(x,\varepsilon )\cap M=C,~\{x_n\in C~|~n>N\}$
\subsection{定理:$M$是闭集,当且仅当:若$\exists \{x_n\}\subset M$,~$x_n\rightarrow x$则$x\in M$}\textcolor{blue}{也就是说闭集就是说有收敛序列的极限都在其内则为闭集。}
\begin{itemize}
	\item “$\Rightarrow$ ”已知$M$是闭集,即证:若$\exists \{x_n\}\subset M$,~$x_n\rightarrow x$则$x\in M$由上一个定理得知$x\in \overline M$而当M为闭集时,M等于闭包。
	\item “$\Leftarrow$” 也是显然因为由于上一个定理得知若$\exists \{x_n\}\subset M$,~$x_n\rightarrow x$则$x\in \overline M$,又因为根据条件得知$x\in M$所以$\overline M\subset M$所以M为闭集。
\end{itemize}
\subsection{子空间$(X,d)$下$M\in X,(M,d)$是子空间}
\subsection{定理:如果大空间是完备的,那么小空间完备的充要条件是小空间是闭集}	证明:\begin{itemize}
	\item "$\Rightarrow$":已知小空间M是闭集,即证:其中的任何柯西列都收敛在其中。
	因为任何M中的柯西序列$\{x_n\} $都是大空间X中的收敛序列。所以其收敛至x。又因为任意$B(x,\varepsilon )\cap M\neq \phi$所以$x\in M'$又因为M是闭集所以$\{x_n\} $收敛。
	\item "$\Leftarrow$":如果小空间M是完备的,其中任意一个柯西序列都是收敛序列也都收敛在其中所以其为闭集。
\end{itemize}
\subsection{定理:连续性用序列定义}
描述:$T(X,d_x)\rightarrow(Y,d_y)$如果T在$x_0$处连续则满足:若$x\rightarrow x_0$则$T(x)\rightarrow T(x_0)$。
\begin{itemize}
	\item "$\Leftarrow$":下证连。续若不连续,则存在$\varepsilon>0$使得对于$\forall \delta=\frac{1}{n}$。存在$x\in B(x,\frac{1}{n}),x\rightarrow x_0 ,$但是$d(T(x),T(x_0))>\varepsilon$所以不满足连续,所以$T(x)\rightarrow T(x_0)$。
	\item "$\Rightarrow$"$\forall \varepsilon  \exists N$满足$n>N,~d(x_n,x)<\delta,\delta=\delta(\varepsilon,N )$因为连续说以存在$\delta(\varepsilon,N ),x\in B(x_0,\delta ),T(x)\in B(T(x_0),\varepsilon )$,这正好是$T(x)\rightarrow T(x_0)$
\end{itemize}

\section{完备化}
\paragraph{完备化定义:}顾名思义就是在空间中添加一些元素是空间变为完备空间。
\paragraph{完备空间的理解:}借助柯西序列与收敛序列来说明,任何收敛序列都是柯西序列,因为柯西序列的极限点不一定在空间中,只要保证柯西序列的极限点在空间中的空间就是完备空间。
\paragraph{定义:等距映射}$(X,d),(\tilde{X},\tilde{d} )$中:
\begin{equation*}
	\begin{cases}
		T:X\rightarrow \tilde{X}\\
		\forall x,y \in X, ~d(x,y)=\tilde(T(x),T(y))
	\end{cases}
\end{equation*}
则称其为等距映射,如果T还是一一映射则称$(X,d),(\tilde{X},\tilde{d} )$为等距空间。
\subsection{定理:}$(X,d)$是不完备的度量空间,那么一定存在一个完备度量空间$(\hat X,\hat d )$其稠密子空间$(W,\hat d),~\overline{W}=X$与$(X,d)$是等距的。
\paragraph{例子}比如有理数在$\mathbb{R}$是稠密的,其加上所有无理数之后就是完备的了。
\pagebreak
\part{巴拿赫不动点定理}
\setcounter{section}{0}
\section{压缩映射}
\paragraph{定义:}$(X,d)$度量空间。\\
$T:X\rightarrow X,~d<1$\\
若存在$0<\alpha <1$使得:\\
\begin{equation*}
	d(Tx,Ty)\leq \alpha~d(x,y)
\end{equation*}
则称T是压缩映射。
\subsection{定理:压缩映射是连续的}
\subsection{压缩映射例子:}
$f:~\mathbb{R}\rightarrow \mathbb{R}~(x\rightarrow f(x)) $
\begin{align*}
	|f(x)-f(y)|=d(f(x),f(y))
\end{align*}\\
一定存在~$d(f(x),f(y))=|f'(x)|d(x,y)$\\
若$\forall\xi, f'(\xi)=\alpha\leq 1$则f一定是压缩的。
\section{不动点定理:}
\large \paragraph{描述:}假设$(X,d)$非空且完备。\\
\begin{equation*}
	T:X\rightarrow X  \mbox{是压缩的}
\end{equation*}
则T有且仅有一个不动点$(T(x_0)=x_0$为不动点)
\paragraph{进一步:}任取一点$x_0\in X$,找一串序列。\\
\begin{align*}
	&x_1=Tx_0\\
	&x_2=Tx_1=T^2x_0\\
	\vdots\\
	&x_n=T^nx_0\\
	&\{x_n\}\rightarrow \overline{x}\mbox{为唯一的不动点}
\end{align*}
\paragraph{证明:}
任取一点$x_0\in X$\\
作$\{x_n\}$,$x_n=T^nx_0$即$x_{n+1}=Tx_n$\\
下证:$\{x_n\}$是柯西序列\\
\begin{align*}
	d(x_{m+1},x_{m})=d(Tx_{m},Tx_{m-1})\leq \alpha d(x_{m+1},x_{m})
	\\
	\leq \alpha^m d(x_1,x_0)
\end{align*}
对于$\forall n>m$
\begin{align*}
	&d(x_n,x_m)\\
	\leq & d(x_n,x_{n-1})+d(x_{n-1},x_{n-2})+...+d(x_1,x_0)
\end{align*}
下证:对$\forall \varepsilon>0,~\exists N,\forall n>m>N,d(x_n,x_m)\leq \varepsilon$
\begin{align*}
	&d(x_{n},x_{m})\\
	\leq &d(x_{n},x_{n-1})+d(x_{n-1},x_{n-2})+...+d(x_{m+1},x_{m})\\
	\leq & (\alpha^{n-m}+\alpha^{n-m-1}+...+\alpha+1)d(x_{m+1},x_{m})\\
	\leq &\frac{\alpha^m(1-\alpha^{n-m} )}{1-\alpha}d(x_{1},x_{0})
\end{align*}
即已知$d(x_n,x_m)\leq \varepsilon$找出那个N。
\begin{align*}
	d(x_n,x_m)\leq \varepsilon\\
	\Rightarrow \frac{\alpha^m(1-\alpha^{n-m} )}{1-\alpha}d(x_{1},x_{0})\leq \varepsilon\\
	\Rightarrow m>log_d(\frac{\varepsilon}{d(x_1,x_0)}(1-\alpha) )
\end{align*}
所以$N=[log_d(\frac{\varepsilon}{d(x_1,x_0)}(1-\alpha) )]+1$。\\
下证:$T\overline{x}=\overline{x}$
\begin{align*}
	d(T\overline{x},\overline{x} )\leq &d(T\overline{x},x_m )+d(x_m,\overline{x} )\\
	\leq &\alpha d(\overline{x},x_{m-1})+d(x_m,\overline{x} )\\
	limt~x_m\rightarrow &\overline{x}:d(\overline{x},x_{m-1})=0\\
	thus:&~d(T\overline{x},\overline{x} )\leq 0
\end{align*}
所以:$d(T\overline{x},\overline{x} )= 0$
所以$T\overline{x}=\overline{x}$。\\
下证唯一:\\
反设不然:
$\exists Ty_1=y_1,Ty_2=y_2$\\
\begin{align*}
	d(y_1,y_2)=d(Ty_1,Ty_2)\leq \alpha d(y_1,y_2)\\
	(1-\alpha)d(y_1,y_2)\leq 0\\
	\Rightarrow (1-\alpha )d(y_1,y_2)=0\\
	\Rightarrow d(y_1,y_2)=0
\end{align*}
所以$y_1=y_2$
\pagebreak
\section{不动点定理的应用:}
\subsection{牛顿法解方程}
牛顿法解方程是通过下面这个序列,其极限是方程的解$f(\overline{x})=0$\\

\paragraph{条件} 序列$x_{n+1}=g(x_n)=x_n-\frac{x_0}{f'(x_0)}$\\
在$[a,b]$上f为两次连续可微$f''$为连续即$f\in C^2[a,b]$,~$\overline{x}$为$f(\overline{x} )=0$的单零点$f'(\overline{x} )\neq 0$那么$g(x)$在这个邻域$[\overline{x}-\delta,\overline{x}+\delta ]$上是压缩的。\\

证明:只需证明:$|g'(x)\leq\alpha<1|$\\
\begin{align*}
	g'(x)&=1-\frac{f'(x)^2-f(x)f''(x)}{f'(x)^2}\\
	&=\frac{f(x)f''(x)}{f'(x)^2}
\end{align*}
因为f为两次连续可微$f''$为连续即$f\in C^2[a,b]$,~$\overline{x}$为$f(\overline{x} )=0$的单零点,所以$lim_{x\rightarrow \overline{x}}g'(x)=0$,并且原函数连续。\\
所以在邻域$[\overline{x}-\delta,\overline{x}+\delta ]$上时,则任意选择$\varepsilon\leq \alpha $,都$\exists B(\overline{x},\delta)$满足$g~B(\overline{x},\delta)\subset B(g(\overline x),\varepsilon) $
所以存在$\delta>0$满足$g(x)$在$[\overline{x}-\delta,\overline{x}+\delta ]$内是压缩的。
\subsection{常微分方程}
度量空间$(C[t_0-\beta,t_0+\beta],d)$上解的存在性证明。
\begin{equation}
	\begin{cases}
		x'(t)=f(t,x(t))\\
		x(t_0)=x_0
	\end{cases}
\end{equation}
设$f \mbox{在} \mathbb{R}=\{(t,x)~|~|t-t_0|\leq a,|x-x_0|\leq b \}$连续。且$|f(t,x)|\leq c,(t,x)\in \mathbb{R}$\\
满足李氏条件:$\exists k>0$使得:
\begin{equation}
	|f(t,x)-f(t,y)|\leq k|x-y|
\end{equation}
则存在$[t_0-\beta,t_0+\beta ]$解存在且唯一。$\beta<min\{a,\frac{b}{c},\frac{1}{k} \}$

\paragraph{证明:}
首先:$(C[t_0-\beta,t_0+\beta],d)$
在$d=max_{t\in [t_0-\beta,t_0+\beta]}|f(t)-g(t)|$是完备的。\\
首先$x(t)=x_0$是常函数,显然是属于连续函数度量空间$(C[t_0-\beta,t_0+\beta],d)$的。\\
\\
定义闭球也就是$\overline{B}(x_0,c\beta)=\{f(t)~|~max|f(t)-x_0|\leq c\beta\},$\\
\paragraph{证明闭球是闭集} 反证若不是闭集那么存在收敛序列其极限$\{f_n\}\rightarrow f_0~$$s.t:d(f_0,x_0)>r$\\
又因为:$d(f_0,x_0)\leq d(f_n,x_0)+d(f_n,f_0)$\\
即$d(f_n,f_0)\geq d(f_0,x_0)- d(f_n,x_0)\geq d(f_0,x_0)-r=e$又因为e是一个常数,所以与$\{f_n\}\rightarrow f_0$矛盾。
 取$X=\overline{B}(x_0,c\beta),\because c\beta <b,\overline{B}(x_0,c\beta)\subset (C[t_0-\beta,t_0+\beta],d)$那么X也是完备的。\\
\paragraph{定义:映射T}\begin{equation*}
	Tx(t)=x_0+\int_{t_0}^{t}f(t,x(t))ds
\end{equation*}
下证:对于任意$Tx(t)\in X$。\\
\begin{align*}
	d(x_0,Tx(t))&=\mathop{max}_{t\in |t-t_0|\leq \beta}\int_{t_0}^{t}f(t,x(t))ds\\
		&\leq c|x-x_0|=c\beta
\end{align*}
所以$Tx(t) \in \overline{B}(x_0,c\beta) $
\paragraph{证明该映射是压缩的}即证:
\begin{equation}
	d(Tx(t),Ty(t))\leq \alpha ~d(x(t),y(t));~x(t),y(t)\in  X
\end{equation}
\begin{align*}
	d(Tx(t),Ty(t))&=\mathop{max}_{t\in |t-t_0|\leq \beta}|Tx(t)-Ty(t)|\\
	&=\mathop{max}_{t\in |t-t_0|\leq \beta}|x_0+\int_{t_0}^{t}f(t,x(t))ds-x_0-\int_{t_0}^{t}f(t,y(t))ds|\\
	&=\mathop{max}_{t\in |t-t_0|\leq \beta}|\int_{t_0}^{t}f(t,x(t))-f(t,y(t))ds|\\
	\mbox{根据李普希兹条件}&\leq \mathop{max}_{t\in |t-t_0|\leq \beta}|\int_{t_0}^{t}k|x(t)-y(t)|ds|\\
	&\leq \mathop{max}_{t,s\in |t-t_0|\leq \beta}k|t-t_0||x(s)-y(s)|\\
	&\leq k\beta d(x(t),y(t))
\end{align*}
又因为$\beta<\frac{1}{k}$所以$d(Tx(t),Ty(t))\leq \alpha ~d(x(t),y(t)),~\alpha\in [0,1)$
所以为压缩的。
\paragraph{下证该序列的收敛点为微分方程的解}设收敛点为:
\begin{equation}
	T\overline{x}(t)=x_0+\int_{t_0}^{t}f(t,\overline{x}(t))ds=\overline{x}(t)
\end{equation}
对等式两边求导可得满足原微分方程。

%-----------------结束-----------
\end{CJK*}
\end{document}
